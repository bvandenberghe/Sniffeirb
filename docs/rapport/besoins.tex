\section{L'état de l'art}
%notamment le logiciel qui est sensé faire pareil que nous et bien sur tcpdump et wireshark JUSTNIFFER
Il existe un grand nombre de logiciels et de librairies qui proposent des fonctionnalités de captures réseaux de bas niveau et d'analyse de paquets.
Nous pouvons notamment citer Libpcap, Scapy, Winpcap ou encore Tcpdump. Ces outils ont été des modèles d'inspiration pour réaliser notre propre \textit{Sniffer}\footnote{Terme anglicisé désginant un dispositif permettant d'analyser le trafic d'un réseau.} réseau.\\

\section{Nos objectifs}
\subsection{Premières étapes}
L'objectif premier était tout d'abord de retrouver les fonctionnalités principales de Wireshark, comme par exemple afficher une liste des paquets circulant sur le réseaux ainsi que les informations s'y rapportant (timestamp, adresse IP source, adresse IP destination, port, protocole...), tout en apportant une solution différente et plus simple d'utilisation à un utilisateur néophyte. Ainsi nous avons défini trois premiers objectifs :
\begin{itemize}
\item Récupérer les communications\footnote{Une communication est définie par un échange réseau entre deux paires (Adresse IP ; Port).} en clair d'un réseau.
\item Stocker les paquets de cette communication.
\item Afficher les différents flux\footnote{Un flux est défini comme un assemblage possible des paquets d'une communication.} possibles de cette communication.\\
\end{itemize}

\subsection{La valeur ajoutée de Sniff'eirb}

Le but de ce projet n'étant pas de reproduire un Wireshark, nous devions y apporter des atouts significatifs. Plusieurs fonctionnalités ont donc été implémentées :

\begin{description}
\item[L'arbre des possibilités pour les flux TCP :] Lors de la capture des paquets circulant sur le réseau pour une même communication, des événements inatendus, comme la répétition de paquets ou encore la perte de certains,  pouvaient survenir. La capture de tous les paquets permet donc de connaître les différentes façons de les réassembler pour obtenir un flux décrivant la communication. 
\item[La gestion du protocole HTTP :] La reconstruction des pages webs visitées par un utilisateur sur le réseau était le fil conducteur du projet. L'analyse des paquets circulant sur le réseau était la première étape afin d' arriver à cette fin.\\
\end{description}

\subsection{Les idées abandonées ou non implémentées}
Beaucoup d'idées durent être abandonnée du fait d'un manque de temps pour les traiter. Voici une liste non exhaustive de ces idées :\\
\begin{itemize}

  \item[\textbf{Récupérer les cookies d'un utilisateur du réseau}] afin de les réinjecter dans le navigateur et pouvoir accéder aux données personnelles de cet utilisateur.\\
  \item[\textbf{Établir un mode \emph{ghost}}] qui permettrait au Sniffer d'être quasiment indétectable sur le réseau. \\
  \item[\textbf{Permettre le choix de l'interface réseau}] pour pouvoir écouter sur un réseau choisi. \\
  \item[\textbf{Exporter les données au format \emph{pcap}}] afin de pouvoir réutiliser ces données avec d'autres analyseurs réseau.\\
  \item[\textbf{Supporter l'analyse des paquets transmis en UDP}] pour récupérer toujours plus d'informations qui transitent sur le réseau.\\
\end{itemize}
