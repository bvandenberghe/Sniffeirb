Dans le cadre du projet de sécurité, nous avions le choix sur notre sujet. En parallèle de ce projet, nous travaillions sur un projet de développement système et réseaux dont le but était la réalisation d'un Sniffer intelligent\footnote{La description des besoins et fonctionnalités plus loin dans le rapport définit cette notion pour le moment ambigüe d'\emph{intelligent}.}. L'idée nous est alors venue de fusionner les deux projets pour n'en faire qu'un. En effet, l'élaboration d'un Sniffer faisait partie des possibilités de sujet du projet de sécurité. Le projet de sécurité était donc d'ajouter à ce Sniffer des fonctionnalités intéressantes dans le but à la fois de le rendre plus fin pour une éventuelle attaque d'un réseau et à la fois de grossir sa truffe pour lui permettre de renifler une piste d'attaque du réseau pour mieux se défendre. Pour confirmer (ou infirmer) l'efficacité de ce Sniffer, nous avons du réalisé plusieurs tests d'analyse de réseaux. Néanmoins, ces analyses ont uniquement été faites à but pédagogique\footnote{Nous certifions donc sur notre honneur ne jamais avoir eu d'idées malveillantes en testant notre Sniffer.}.\\

 Ce rapport décrit donc dans un premier temps les besoins et l'existant, puis l'architechture envisagée de notre Sniffer, et enfin son implémentation effectivement réalisée.
